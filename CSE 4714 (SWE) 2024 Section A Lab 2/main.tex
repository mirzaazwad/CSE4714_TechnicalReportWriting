\documentclass[12pt, a4paper]{article}
\usepackage[margin=1in]{geometry}
\usepackage{stix2}
\usepackage[colorlinks]{hyperref}

\usepackage{amsmath, amssymb}

\title{CSE 4714 (SWE) 2024 Section A Lab 2}
\author{Mohammad Ishrak Abedin}
\date{November 2024}

\begin{document}

\maketitle

\section{Quotation Marks}

The rude teacher said, ``You cannot do the lab with any group you want without prior permission.''

\section{Mathematical Expressions}
There are two modes, inline and display maths.

\subsection{Inline Math}
We can put a mathematical expression like $(a + b)^2 = a^2 + 2ab + b^2$. We can also write it as \( (a + b)^2\qquad = a^2 + 2ab + b^2  \). There is one more way, 
\begin{math}
    (a + b)^2 = a^2 + 2ab + b^2
\end{math}.

\subsection{Display Math} \label{ssec:displaymath}
We can write display math as, \[ (a + b)^2 = a^2 + 2ab + b^2 \]. We can also write it as, \begin{displaymath}
    (a + b)^2 = a^2 + 2ab + b^2
\end{displaymath}
Finally, we can also write it as,
\begin{equation}
    (a + b)^2 = a^2 + 2ab + b^2
\end{equation}

We also have,
\begin{equation} \label{eqn:absq}
    a ^ 2 - b ^ 2 = (a + b)(a - b)
\end{equation}
\begin{equation}
    ab = ba
\end{equation}

In \autoref{eqn:absq}, we can see the formula for $a^2 - b^2$. We can learn about displaymath in \autoref{ssec:displaymath}.

\subsection{Super and Subscript}
\begin{equation}
    a^{3.14159265} + b_{(i, j, k)} + C^{x}_{yz}
\end{equation}

\subsection{Fractions, Operators, ...}
\begin{equation*}
    2 \times 5 = 10
\end{equation*}


\begin{equation}
    \sqrt{\frac{(a + b)}{(a - b)}} = \frac{1}{2} = \sqrt{x + y}
\end{equation}

Fractions will change their display inline, $\displaystyle \frac{(a + b)}{(a - b)} = \frac{1}{2}$. Fractions will change their display inline, $\frac{(a + b)}{(a - b)} = \frac{1}{2}$. Fractions will change their display inline, $\frac{(a + b)}{(a - b)} = \frac{1}{2}$. Fractions will change their display inline, $\frac{(a + b)}{(a - b)} = \frac{1}{2}$. Fractions will change their display inline, $\frac{(a + b)}{(a - b)} = \frac{1}{2}$. Fractions will change their display inline, $\frac{(a + b)}{(a - b)} = \frac{1}{2}$. 

\subsection{Operators and Comparison}
\begin{equation}
    1 \times 2 = 2 = 1 + 1 = 3 - 1 = 4 / 2  
\end{equation}

% Following example is wrong
% \begin{equation}
%     sin^2\theta + cos^2\theta = 1
% \end{equation}

\begin{equation}
    \sin^2\theta + \cos^2\theta = 1 = \tan^{-1}\alpha
\end{equation}

\subsection{Bracketization}
\begin{equation}
    \left.
    \sqrt{\frac{(a + b)}{\sqrt{\frac{(a + b)}{(a - b)}}}} + 2
    \right\}
    \times 5 = \frac{1}{2} = \sqrt{x + y}
\end{equation}

\subsection{Greek Alphabets}
\begin{equation}
    \alpha \beta \gamma \delta = A B \Gamma \Delta
\end{equation}

\subsection{Calculus}
We can write limits.

\begin{equation}
    \lim_{x \rightarrow \infty} \frac{1}{x} = 0
\end{equation}
\begin{equation}
    \int_{x = 0}^{100} x dx = 500 = \sum_{i = 5}^{100} x_i = \prod_{\theta = 5^{\circ}}^{\pi} x_{\theta}
\end{equation}
\begin{equation}
    \int\limits_{x = 0}^{100} x dx = 500
\end{equation}

\subsection{Calligraphy}

\begin{equation}
    \mathcal{ABCDE}\mathfrak{ABCDE}\mathbb{ABCDE}
\end{equation}

\subsection{Set Operations}

\begin{equation}
    A \cup B = C \cap D\ \text{for all}\ x \in \mathbb{Z}
\end{equation}

\subsection{Vector Operations}
\begin{equation}
    \Vec{V} = \Vec{A} \times \Vec{B}
\end{equation}

\begin{equation}
    \Vec{A} \cdot \Vec{B} = 5
\end{equation}

\begin{equation}
    \hat{i} \times \hat{j} = \hat{k}
\end{equation}

\begin{equation}
    1 = 1 < 2 > 5 \neq 10 \geq 2 \leq 5
\end{equation}

\subsection{AMS Math Environments}

\begin{equation*}
    a + b = 5
\end{equation*}


\begin{multline}    
    a + b + c + a + b + c + a + \\
    b + c + a + b + c + a + b + c + \\
    a + b + c + a + b + c + a + b + c + \\
    a + b + c + a + b + c + a + b + c + a + \\
    b + c + a + b + c + a + b + c + a + b + \\
    c + a + b + c = 10
\end{multline}

\begin{equation}
    \begin{split}
         x &= \frac{10}{20} \\
         &= \frac{1}{2} \\
         &= 0.5
    \end{split}
\end{equation}

\begin{align}
    a + 2b - 3c &= 10 \\
    2a + 5b + 6c &= -2 \\
    -a + b + 6c &= -4
\end{align}

\begin{gather*}
    2b - 3c = 10 \\
    2a = -2 \\
    -a + b + 6c = -4
\end{gather*}

\subsection{Matrices}

\begin{equation*}
    \mathbf{A} = \begin{vmatrix}
        1 & 2 & 33 \\
        4 & 5 & 6 \\
        4 & 5 & 6 \\
    \end{vmatrix}
\end{equation*}

\end{document}
