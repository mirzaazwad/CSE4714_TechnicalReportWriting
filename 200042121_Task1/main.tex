\documentclass[12pt, a4paper]{article}

\usepackage[margin=0.65in]{geometry}

\usepackage{parskip}
\usepackage{imfellEnglish}
\usepackage{kantlipsum}
\usepackage[hidelinks]{hyperref}
\usepackage[20pt]{multicol}
\usepackage{ragged2e}
% ragged2e contributed to the beautifully centered “A Small Story” section
% • Try to figure out how the code-style or teletype fonts are printed

\title{The Art of Espresso Making}
\author{Mirza Mohammad Azwad}
\date{\today}
\begin{document}

\begin{titlepage}
    \maketitle
\tableofcontents
\end{titlepage}

\begin{abstract}
    In this document, we explore the nuances of crafting the perfect espresso, from selecting the right
beans to mastering the brewing process. Whether you’re a seasoned barista or a coffee enthusiast, this
guide will deepen your appreciation for the art of espresso making.
\end{abstract}

\section{Introduction}
Espresso, derived from the Italian word meaning ``pressed out", is a concentrated form of coffee known for its rich flavor and velvety crema. In this guide, we delve into the intricacies of creating this beloved
beverage.
\section{Selecting the Beans}
The foundation of any great espresso lies in the quality of the beans. When selecting beans for espresso, consider the following factors:
\subsection{Origin}
Different regions produce beans with distinct flavor profiles. Experiment with beans from various origins to discover your preferences.
\subsection{Roast Level}
Espresso beans are typically roasted to a dark or medium-dark level to bring out their bold flavors. However, the ideal roast level may vary based on personal taste.
\subsection{Freshness}
Freshly roasted beans retain their flavors better than those that have been sitting on the shelf for weeks. Look for beans with a roast date to ensure freshness.
\section{The Brewing Process}
\begin{multicols}{2}
    \subsection{Grinding the Beans}
    Grind the beans to a fine consistency just before brewing to preserve their flavors. The two most popular ways to grind coffee beans are with a burr grinder or blade grinder.
    \subsection{Tamping}
    Evenly distribute the ground coffee in the portafilter and tamp it down firmly to create a uniform surface.
    \subsection{Extraction}
    Attach the portafilter to the espresso machine and initiate the brewing process. The water should be heated to around 195-205$^{\circ}$F for optimal extrac-
tion.
\subsection{Timing}
Monitor the extraction time closely. A typical
shot of espresso should take approximately 25-30
seconds to brew.
\end{multicols}
\section{Enhancing The Experience}
The art of espresso making extends beyond the brewing process. Consider the following tips to enhance your espresso experience:
\subsection{Serving Temperature}
Serve espresso in preheated cups to maintain its temperature and preserve its flavor.
\section{Conclusion}
Mastering the art of espresso making requires dedication, practice, and a keen appreciation for the nuances of coffee. With the right beans, equipment, and technique, you can create espresso that delights the senses and leaves a lasting impression.
\section{A Small Story}
\begin{multicols}{2}
    \FlushRight
In a quaint cafe, where the world seemed to slow,
A tale of coffee began to softly flow.
Amidst the aroma of beans freshly ground,
Lay secrets and stories waiting to be found.
In the heart of the city, where dreams took flight,
A weary traveler sought solace in the night.
With a cup of dark brew, steaming and bold,
Memories and musings began to unfold.
Each sip held a moment, a tale to tell,
Of love lost and found, in a caf ́e’s spell.
The first taste of morning, a kiss of the sun,
Awakening senses, a journey begun.
In the corner, a poet with pen in hand,
Crafted verses inspired by coffee’s command.
Each word danced on paper, fueled by caffeine,
A symphony of thoughts, a writer’s dream.
From distant lands, where the beans were born,
To the hands of baristas, each cup adorned.
A journey of flavor, a story untold,
In every rich brew, a narrative unfolds.
So raise your cup high, let the stories flow,
In the warmth of coffee, let your imagination grow.
For in every sip, a world awaits,
A story of coffee, in its essence, vibrates.
\end{multicols}
\end{document}
